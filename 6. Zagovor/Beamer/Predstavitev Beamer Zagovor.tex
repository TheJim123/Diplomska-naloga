\documentclass[t, 10pt]{beamer} % [t] pomeni poravnavo na vrh slida

% \usepackage{etex} % vključi ta paket, če ti javi napako, da imaš naloženih preveč paketov.

% standardni paketi
\usepackage[utf8]{inputenc}
\usepackage[T1]{fontenc}
\usepackage[slovene]{babel}
\usepackage{lmodern}

\usepackage{amsfonts,url, amsthm}
\usepackage{palatino}
\usepackage{graphicx, tikz, array}
\usepackage{stmaryrd}

\renewcommand\textbullet{\ensuremath{\bullet}}  % to get rid of font warnings

% podatki
\title{Linearna algebra nad polkolobarji}
\author{Jimmy Zakeršnik}
\institute{mentor: prof. dr. Tomaž Košir}

% tvoj izbran stil predstavitve
\usefonttheme{professionalfonts}
\mode<presentation>
{
	\usetheme{Berlin}
	\usecolortheme{default}
	\useoutertheme{infolines}
	\useinnertheme[shadows]{rounded}
}
\setbeamertemplate{footline}{}            % pri nekaterih temah je potrebno odstraniti
\setbeamertemplate{navigation symbols}{}  % nogo, v tem primeru to odkomentiraj

\newtheorem{trditev}{Trditev}
\newtheorem{izr}{Izrek}

\newcounter{defcount}
\newcounter{opombe}
\newcounter{zgledcount}

\newenvironment{opomba}{\begin{flushleft}\stepcounter{opombe}\textbf{Opomba \arabic{opombe}:}}{\hfill\end{flushleft}}
\setlength{\parindent}{0mm}

\newenvironment{zgled}{\begin{flushleft}\stepcounter{zgledcount}\textbf{Zgled \arabic{zgledcount}:}}{\hfill\end{flushleft}}
\setlength{\parindent}{0mm}

\newenvironment{definicija}{\begin{flushleft}\stepcounter{defcount}\textbf{Definicija \arabic{defcount}:}}{\hfill\end{flushleft}}
\setlength{\parindent}{0mm}

\newcommand{\abs}[1]{\ensuremath{\lvert #1 \rvert}}
\newcommand{\mth}[1]{\ensuremath{\mathbb{#1}}}
\newcommand{\R}{\mth{R}}
\newcommand{\Pplus}[1]{\mth{#1}^{+}}
\newcommand{\Z}{\mth{Z}}
\newcommand{\N}{\mth{N}}
\newcommand{\No}{\mth{N}_0}
\newcommand{\C}{\mth{C}}
\newcommand{\Qo}{\mth{Q}_0}
\newcommand{\pojem}[1]{\emph{#1}}
\newcommand{\con}{\ensuremath{\mathscr{C}}}
\newcommand{\padex}[2]{\ensuremath{{#1}^{\underline{#2}}}}
\newcommand{\rastx}[2]{\ensuremath{{#1}^{\bar{#2}}}}
\newcommand{\map}[3]{\ensuremath{{#1}~: {#2} \rightarrow {#3}}}
\newcommand{\pra}[3]{{#1}{\ast}({#2}) = {#3}}
\newcommand{\Gen}[1]{\ensuremath{\left<{#1}\right>}}

%  ukaz za počrnitev enega slida. Nariše črn pravokotnik višine #1 na dno.
\newcommand{\fillblack}[1]{
\begin{tikzpicture}[remember picture, overlay]
    \node [shift={(0 cm,0cm)}]  at (current page.south west)
        {%
        \begin{tikzpicture}[remember picture, overlay] at (current page.south west)
            \draw [fill=black] (0, 0) -- (0,#1 \paperheight) --
                              (\paperwidth,#1 \paperheight) -- (\paperwidth,0) -- cycle ;
        \end{tikzpicture}
        };
        \draw (current page.north west) rectangle (current page.south east);
\end{tikzpicture}
}

% Projektor, ki sveti na tablo je zoprn, ker zavzema prostor za pisanje.
% Praktična rešitev je, da se platno spusti samo do table, spodnji del
% predstavitve pa naredimo popolnoma črn. Tako projektor na tablo ne projecira
% svetlobe in je pisanje nemoteno s strani predstavitve, ki je cela nad tablo.

% Za to da dobite spodaj zatemnjen slide, je potrebno na konec frame-a dodati
% ukaz \fillblack{delež}, kjer je delež številka med 0 in 1, ki pove, kakšen
% delež slida želite imeti zapolnjen s črnim pravokotnikom. Spodaj so 4 slidi,
% ki imajo začrnjenih: 33%, 50%, 33% in 20%.  Za šolske table v 2.02 in 2.03 je
% 33% ravno prav, da lahko pišete po celi tabli.

% Črni pravokotnik ne zavzema prostora na slidu -- preprosto nariše se čez
% spodnjo tretjino (ali kolikor pač želite). To pomeni, da se pokrijve tudi ves
% tekst spodaj. Za to je na začetku dokumenta tudi nastavljena vertikalna
% poravnava na ``zgoraj'', z razliko od običajne sredinske (to je tisti [t]).
% Tako se ves tekst obdrži na vidnem delu slida, dokler je to možno. Tretji
% slide spodaj demonstrira, kaj se zgodi v primeru preveč besedila.

% Kod demonstrira zadnji slide, je vseeno, kjer napišemo ukaz, toda ponavadi ga
% napišemo na dnu, da se izognemo morebitnim nevšečnostim ali neželenemu
% spacingu. Pri nekaterih temah se izriše tudi noga, ki se izriše na koncu
% slida, torej po tem ko je naš pravokotnik že narisan, kar zna biti moteče.
% To se na primer zgodi tudi pri standardni temi Warsaw. Rešitev je, da nogo
% odstranimo (saj bi bila itak prekrita). V preambuli so zakomentirani ukazi,
% kako to naredimo.

\begin{document}

\begin{frame}
  \maketitle
  \fillblack{0.3}
\end{frame}

\begin{frame}{Napovednik}
	\begin{enumerate}
		\item Motivacija
		\item Monoidi in urejenost
		\item Polkolobarji in dioidi
		\item Polmoduli in moduloidi
		\item Matrike nad polkolobarji
		\item Pideterminanta in karakteristični pipolinom
		\item Posplošeni Cayley-Hamiltonov izrek
		\item Literatura
	\end{enumerate}

	\fillblack{0.3}
\end{frame}

\begin{frame}{Motivacija}
  Polkolobarjev je veliko - pojavljajo se v skoraj vsakem področju matematike. Nekateri primeri so:
  \begin{itemize}
    \item $\No$ oz. $\Pplus{Z}$, $\Pplus{Q}$, $\Pplus{R}$ za standardne operacije $+$ in $*$,
    \item \pojem{max-plus algebra}~$(\R\cup\{-\infty\}, max, +)$ in \pojem{min-plus algebra}~$(\R\cup\{\infty\}, min, +)$,
    \item Boolove algebre,
    \item Potenčne množice za $\cup$ in $\cap$.
  \end{itemize}

  \fillblack{0.3}
\end{frame}

\begin{frame}{Monoidi in urejenost}

\begin{block}{Definicija}<1->
	Komutativen monoid $(M, *)$ je \pojem{kanonično urejen}, če je kanonična šibka urejenost $x\leq y \iff \exists z\in M: y = x * z$ na $M$ antisimetrična.
	\end{block}
\begin{block}{Izrek}<2->
Monoid ne more hkrati biti grupa in kanonično urejen.
\end{block}
\begin{block}{Izrek}<3->
Naj bo monoid  $(M, *)$ okrajšljiv in naj zadošča pogoju pozitivnosti. Potem je kanonična šibka urejenost $\leq$ na $M$ antisimetrična.
\end{block}
\fillblack{0.3}
\end{frame}

\begin{frame}{Polkolobarji in dioidi}
	
	\begin{block}{Definicija}
	Za neprazno množico $R$, ki je opremljena z operacijama $\oplus$ in $\otimes$ pravimo, da je \pojem{polkolobar}, če zanjo velja naslednje:
		\begin{enumerate}
			\item $(R, \oplus)$ je komutativen monoid z nevtralnim elementom $0$,
			\item $(R, \otimes)$ je monoid z enoto $1$,
			\item (leva in desna) distributivnost $\otimes$ in $\oplus$,
			\item $0 \otimes a = 0 = a\otimes 0; \forall a\in R$.
		\end{enumerate}
	Če je operacija $\otimes$ komutativna, pravimo, da je polkolobar $R$ komutativen. \\ Oznaka: $(R, \oplus, \otimes)$.
	\end{block}

  \fillblack{0.3}
\end{frame}

\begin{frame}{Polkolobarji in dioidi}
	\begin{block}{Definicija}<1->
	Naj bo $(R, \oplus, \otimes)$ polkolobar. Če je $(R, \oplus)$ kanonično urejen monoid, pravimo, da je $(R, \oplus, \otimes)$ \pojem{dioid}.
	\end{block}

	\begin{block}{Trditev}<2->
	Če je $(R, \oplus, \otimes)$ polkolobar na katerem kanonična šibka urejenost $\leq_\oplus$ ni antisimetrična in je $\mathcal{E}$ ekvivalenčna relacija s predpisom $x\mathcal{E} y \iff x\leq y ~\& ~y \leq x$, je $R/\mathcal{E}$ dioid za inducirani operaciji.
	\end{block}
 \fillblack{0.3}
\end{frame}

%\begin{frame}{Polkolobarji in dioidi}
%	
%	\begin{block}{Definicija}
%	Dioid $(R, \oplus, \otimes)$ je \pojem{poln}, če je za kanonično delno urejenost definirano preko $\oplus$ poln kot množica in ustreza t.~i.~posplošeni distributivnosti: $\forall P \subseteq R, \forall r \in R:$ \begin{itemize}
%\item $\left(\bigoplus_{p \in P} p\right) \otimes r = \bigoplus_{p \in P} \left(p\otimes r\right)$
%\item $r \otimes \left(\bigoplus_{p \in P} p\right) = \bigoplus_{p \in P} \left(r\otimes p\right)$ 
%\end{itemize}
%	\end{block}
%  \fillblack{0.3}
%\end{frame}

\begin{frame}{Polmoduli in moduloidi}
	\begin{block}{Definicija}
			Naj bo R polkolobar. \pojem{Levi~R-polmodul} je komutativen monoid $(M, +)$ z aditivno identiteto $\theta$, na katerem imamo definirano množenje s skalarjem \map{\cdot}{R\times M}{M}, ki zadošča naslednjim pogojem za vsaka $\lambda,\mu\in R$ in vsaka $x, y\in M$:
			\begin{enumerate}
				\item $(\lambda\mu) \cdot x = \lambda \cdot (\mu \cdot x)$,
				\item $\lambda\cdot(x + y) = \lambda\cdot x + \lambda\cdot y$~in~$(\lambda + \mu)\cdot x = \lambda\cdot x + \mu\cdot x$,
				\item $1\cdot x = x$~in~$\lambda\cdot\theta = \theta = 0\cdot x$
			\end{enumerate}
			Če je $R$ dioid in $(M, +)$ kanonično urejen, mu pravimo $R$-moduloid.
	\end{block}
\fillblack{0.3}
\end{frame}

%\begin{frame}{Polmoduli in moduloidi}
%	\begin{block}{Definicija}
%			Naj bo $R$ dioid in $(M, +)$ (levi) $R$-polmodul. Če je $(M, +)$ kanonično delno urejen glede na $+$, mu pravimo (levi) $R$-moduloid.
%	\end{block}
%\fillblack{0.3}
%\end{frame}

\begin{frame}{Polmoduli in moduloidi}
	\begin{block}{Definicija}<1->
			Naj bo $X$ neka neprazna družina elementov $R$-polmodula $(M, +)$. Najmanjši $R$-podpolmodul v $M$, ki vsebuje $X$, imenujemo $R$-polmodul generiran z $X$ in ga označimo z \Gen{X}. Če je \Gen{X} $= M$, pravimo, da $X$ \pojem{generira} $M$. Če je $X$ končna družina, ki generira $M$, pravimo, da je $M$ \pojem{končno generiran}.
	\end{block}
	\begin{block}{Definicija}<2->
			Rang $R$-polmodula $(M, +)$ je najmanjše naravno število $n\in \N$ za katerega obstaja družina $X\subseteq M$ kardinalnosti $n$, ki generira $M$.
	\end{block}
\fillblack{0.3}
\end{frame}

\begin{frame}{Polmoduli in moduloidi}
	\begin{block}{Definicija}<1->
			Naj bo $(M, +)$ $R$-polmodul z enoto $\theta$ in $X = (x_j)_{j\in J}$  neka neprazna (končna ali neskončna) družina elementov iz $M$. Pravimo, da je $X$ \pojem{linearno neodvisna}, če za vsak disjunktni par $I_1, I_2 \subseteq J$ velja \Gen{X_{I_1}}$\cap$\Gen{X_{I_2}} $= \{\theta\}$.
	\end{block}
	\begin{block}{Definicija}<2->
			Naj bo $(M, +)$ $R$-polmodul z enoto $\theta$ in $X = (x_j)_{j\in J}$  neka neprazna (končna ali neskončna) družina elementov iz $M$. Pravimo, da je $X$ \pojem{šibko linearno neodvisna}, če za njo velja pogoj: $\forall j \in J: x_j \notin \Gen{X\setminus\{x_j\}}$.
	\end{block}
\fillblack{0.3}
\end{frame}

\begin{frame}{Polmoduli in moduloidi}
	\begin{block}{Definicija}<1->
			Naj bo $(M, +)$ $R$-polmodul z enoto $\theta$ in $X = (x_j)_{j\in J}$  neka neprazna (končna ali neskončna) družina elementov iz $M$. Pravimo, da je $X$ \pojem{baza} $M$, če je linearno neodvisna in generira $M$.
	\end{block}
	\begin{block}{Definicija}<2->
			Naj bo $(M, +)$ $R$-polmodul z enoto $\theta$ in $X = (x_j)_{j\in J}$  neka neprazna (končna ali neskončna) družina elementov iz $M$. Pravimo, da je $X$ \pojem{šibka baza} $M$, če je šibko linearno neodvisna in generira $M$.
	\end{block}
\fillblack{0.3}
\end{frame}

\begin{frame}{Polmoduli in moduloidi}
	\begin{block}{Definicija}<1->
			Naj bo $(M, +)$ $R$-polmodul z enoto $\theta$ in $X = (x_j)_{j\in J}$  neka neprazna (končna ali neskončna) družina elementov iz $M$. Pravimo, da je $X$ \pojem{prosta baza} $R$-polmodula $M$, če je prosta množica v $M$ in generira cel $M$. Polmodulu, ki premore kako prosto bazo, pravimo \pojem{prosti polmodul}.
	\end{block}
Proste baze so hkrati tudi šibke.
	\begin{block}{Izrek}<2->
			Če $R$-polmodul $M$ premore kako neskončno šibko bazo, so vse šibke baze $M$ neskončne.
	\end{block}
\fillblack{0.3}
\end{frame}

\begin{frame}{Polmoduli in moduloidi}
	\begin{block}{Definicija}
			Naj bo $(M, +)$ $R$-polmodul z enoto $\theta$ in $X = (x_j)_{j\in J}$  neka neprazna družina elementov iz $M$. Pravimo, da je vektor $x$ \pojem{razcepen} na \Gen{X}, če in samo če obstajata taka vektorja $y, z \in \Gen{X}$, oba različna od $x$, da je $x = y + z$. Če $x$ ni razcepen na \Gen{X}, pravimo, da je \pojem{nerazcepen} na \Gen{X}.
	\end{block}
\fillblack{0.3}
\end{frame}

\begin{frame}{Polmoduli in moduloidi}
\begin{block}{Izrek}
			Naj bo $(R, \oplus, \otimes)$ dioid z enotama $0$ in $1$ za katerega velja $r\oplus p = 1 \Rightarrow r = 1 \lor p = 1$ in $r \otimes p = 1 \Rightarrow r = 1 \land p = 1$. Naj bo $(M, +)$~$R$-moduloid na katerem za $y\in M\setminus\{\theta\}, x\in M\setminus\{y\}$ in poljuben $\lambda \in R$ velja $y = \lambda\cdot y + x \Rightarrow \lambda = 1$. Potem velja, da če $M$ premore kako bazo, je ta enolično določena.
	\end{block}
Primer: $(\No, +, *)$
\fillblack{0.3}
\end{frame}

%\begin{frame}{Matrike}
%	
%	\begin{block}{Definicija}<1->
%	Naj bosta $M$ in $N$ $R$-polmodula. Preslikava \map{L}{M}{N} je \pojem{linearna preslikava}, če je aditivna in homogena.
%	\end{block}
%	
%	\begin{block}{Definicija}<2-> 
%		Iz elementov polkolobarja $(R,\oplus, \otimes)$ sestavimo $m\times n$ matrike za poljubna $m,n\in\N$. Pri tem seštevanje matrik definiramo po komponentah, množenje pa na sledeč način za $A\in M_{m\times n}(R), B\in M_{n\times l}(R)$: \\
%	
%	$A * B = C \in M_{m\times l}(R);~ c_{ij} = \sum_{k = 1}^{n}(a_{ik}\otimes b_{kj})$ za $1 \leq i \leq m$ in $1\leq j \leq l$
%	\end{block}
%	\fillblack{0.3}
%\end{frame}

\begin{frame}{Matrike}
	\begin{block}{Izrek}<1->
		Naj bo $R$ komutativen polkolobar ter naj bosta $A, B\in M_n(R)$. Če velja $A*B = I_n$, je tudi $B*A = I_n$.
	\end{block}
	\begin{block}{Trditev}<2->
	Naj bo $R$ komutativen polkolobar v katerem $1 \notin V(R)$ in za $\forall u, v \in R$ velja sklep $1 = u \oplus v \Rightarrow u\in U(R) \lor v\in U(R)$. Naj bo $A\in M_n(R)$. Če za $\forall i, j \in \{1, 2, \ldots n\}$ velja $a_{ii}\in U(R)$ in $a_{ij}\in V(R)$, je matrika $A$ obrnljiva.
\end{block}
	\fillblack{0.3}
\end{frame}

\begin{frame}{Matrike}
	\begin{block}{Definicija}<1->
		Naj bo $M$ končno generiran $R$-polmodul in $T$ ter $S$ neki njegovi šibki bazi. Matriki $A$, ki slika elemente $T$ v elemente $S$ pravimo \pojem{prehodna matrika} iz $T$ v $S$.
	\end{block}
	\begin{block}{Definicija}<2->
		Naj bo $R$ komutativen polkolobar in $A\in M_{n\times m}(R)$. Najmanjšemu številu $k\in \N$, za katerega obstajata matriki $B\in M_{n\times k}(R)$ in $C\in M_{k\times m}(R)$, da je $A = B*C$, pravimo \pojem{faktorski rang} matrike $A$ in ga označimo z $\rho_s(A)$.
	\end{block}
	\fillblack{0.3}
\end{frame}

\begin{frame}{Matrike}
	\begin{block}{Izrek}<1->
		Naj bo $R$ komutativni polkolobar in $M$ $R$-polmodul ranga $r$ s šibkima bazama $S$ in $T$. Potem za vsako prehodno matriko $A$ iz $T$ v $S$ velja $r \leq \rho_s(A)$ in obstaja prehodna matrika $\widehat{A}$ iz $T$ v $S$ za katero je $r = \rho_s(\widehat{A})$.
	\end{block}
	\begin{block}{Trditev}<2->
		Naj bo $R$ komutativen polkolobar in $M$ končno generiran prost $R$-polmodul. Potem za vsako šibko bazo $S$ in za poljubno prosto bazo $T$ velja $\abs{T}\leq \abs{S}$.
	\end{block}
	\fillblack{0.3}
\end{frame}

\begin{frame}{Matrike}
	\begin{block}{Izrek}
		Naj bo $R$ komutativen polkolobar in $M$ prost $R$-polmodul ranga $r$. Naj bo $T$ neka prosta baza $M$. Potem so za šibko bazo $S$ naslednje trditve ekvivalentne: \begin{enumerate}
			\item $S$ je prosta baza $M$
			\item $\abs{S}=r$
			\item prehodna matrika med $T$ in $S$ je enolično določena in obrnljiva
		\end{enumerate}
	\end{block}
	\fillblack{0.3}
\end{frame}

\begin{frame}{Matrike}
	\begin{block}{Definicija}<1->
		Naj bo $R$ polkolobar in $M = R^n$ polmodul nad $R$. Naj bo $A\in M_n(R)$ matrika, ki pripada endomorfizmu \map{h}{M}{M}. Pravimo, da je $\lambda\in R$ \pojem{lastna vrednost} matrike $A$, če obstaja tak $v\in M\setminus\{\theta\}$, da velja $A* v = \lambda \cdot v$. Takemu vektorju $v$, če obstaja, pravimo \pojem{lastni vektor} matrike $A$ za $\lambda$.
	\end{block}
	\begin{block}{Izrek}<2->
		Naj bo $R$ komutativen dioid in $A\in M_n(R)$. Potem je $\lambda$ lastna vrednost matrike $A$ natanko tedaj, ko so stolpci matrike $\bar{A}(\lambda) = \begin{bmatrix}
			A & \lambda\cdot I_n \\
			I_n & I_n
		\end{bmatrix}$ linearno odvisni. 
	\end{block}
	\fillblack{0.3}
\end{frame}

\begin{frame}{Pideterminanta in karakteristični pipolinom}
	\begin{block}{Definicija}
			Naj bo $R$ nek polkolobar in $X\in M_n(R)$. Urejeni dvojici podani s predpisom \begin{align*}
				pdt(X) &= \big(\bigoplus_{\substack{\pi\in P^{+}(n) \\ \sigma\in P(n)}}\sigma(\bar{\pi}(X)), \bigoplus_{\substack{\pi\in P^{-}(n) \\ \sigma\in P(n)}}\sigma(\bar{\pi}(X))\big) \\ &= (\llbracket det^{+}(X) \rrbracket, \llbracket det^{-}(X) \rrbracket) = (pdt^{+}(X), pdt^{-}(X))
			\end{align*} pravimo~\pojem{pideterminanta} matrike $X$.
	\end{block}
	\fillblack{0.3}
\end{frame}

\begin{frame}{Pideterminanta in karakteristični pipolinom}
	\begin{block}{Definicija}
		Naj bo $R$ nek polkolobar in $X\in M_n(R)$. \pojem{Karakteristični pipolinom} matrike $X$ v spremenljivki $\lambda$ je urejena dvojica polinomov, podana s predpisom \begin{align*}
			pp_X(\lambda) &= (\llbracket p_X^{+}(\lambda)\rrbracket, \llbracket p_X^{-}(\lambda)\rrbracket) \\ &= (pp_X^{+}(\lambda), pp_X^{-}(\lambda))
		\end{align*}
		Pri tem polinoma $p_X^{+}(\lambda)$ in $p_X^{-}(\lambda)$ dobimo tako, da se pretvarjamo, da delamo nad poljem in zapišemo pripadajoči karakteristični polinom $p_X(\lambda) = p_X^{+}(\lambda) \ominus p_X^{-}(\lambda)$.
	\end{block}
	\fillblack{0.3}
\end{frame}

\begin{frame}{Posplošeni Cayley-Hamiltonov izrek}
	\begin{block}{Izrek}
		Naj bo $R$ poljuben polkolobar in $X\in M_n(R)$ neka kvadratna matrika nad $R$. Potem je $pp_{X}^{+}(X) =pp_{X}^{-}(X)$.
	\end{block}
	\fillblack{0.3}
\end{frame}

\begin{frame}{Literatura}
	\begin{itemize}
		\item M.~Akian, S.~Gaubert in A.~Guterman,~\emph{Linear independence over tropical semirings and beyond}, v: Tropical and Idempotent Mathematics~vol.~\textbf{495}~(ur.\ G.~Litvinov in S.~Sergeev), Amer. \ Math.\ Soc., Providence, 2008, str.\ 1--38.
		
		\item M.~Gondran in M.~Minoux, \emph{Graphs, dioids and semirings: New models and algorithms}, Operations Research/Computer Science Interfaces \textbf{41}, Springer, Boston, 2008; dostopno tudi na \url{https://www.researchgate.net/publication/266193429_Graphs_Dioids_and_Semirings_New_Models_and_Algorithms}.
		
	\end{itemize}
	\fillblack{0.3}
\end{frame}

\begin{frame}{Literatura}
	\begin{itemize}
		\item R. Grosu, \emph{The Cayley-Hamilton theorem for noncommutative semirings}, v: Implementation and Application of Automata~(ur.\ M.~Domaratzki, K.~Salomaa), Springer, Berlin, 2011, str. 143--153.
		\item C. Reutenauer in H.~Straubing, \emph{Inversion of matrices over a commutative semiring},~Journal of Algebra~\textbf{88}~(1984)~350--360.
		
		\item D.~E.~Rutherford,~\emph{XIX.--The Cayley-Hamilton theorem for semi-rings}, v: Proceedings of the Royal Society of Edinburgh Section A \textbf{66}~(1964)~211--215.
		
		\item Y.\ J.\ Tan,~\emph{Bases in semimodules over commutative semirings}, v: Linear Algebra Appl.~\textbf{443}~(2014)~139--152.
	\end{itemize}
	\fillblack{0.3}
\end{frame}

\end{document}
